%!TEX TS-program = xelatex

% Шаблон документа LaTeX создан в 2018 году
% Алексеем Подчезерцевым
% В качестве исходных использованы шаблоны
% 	Данилом Фёдоровых (danil@fedorovykh.ru) 
%		https://www.writelatex.com/coursera/latex/5.2.2
%	LaTeX-шаблон для русской кандидатской диссертации и её автореферата.
%		https://github.com/AndreyAkinshin/Russian-Phd-LaTeX-Dissertation-Template

\documentclass[a4paper,14pt]{article}

\input{data/preambular.tex}
\begin{document} % конец преамбулы, начало документа
	%\begin{titlepage}
	\begin{center}
 		ФЕДЕРАЛЬНОЕ  ГОСУДАРСТВЕННОЕ АВТОНОМНОЕ \\
		ОБРАЗОВАТЕЛЬНОЕ УЧРЕЖДЕНИЕ ВЫСШЕГО ОБРАЗОВАНИЯ\\
		«НАЦИОНАЛЬНЫЙ ИССЛЕДОВАТЕЛЬСКИЙ УНИВЕРСИТЕТ\\
		«ВЫСШАЯ ШКОЛА ЭКОНОМИКИ»
	\end{center}
	
	\begin{center}
		\textbf{Московский институт электроники и математики}
		
		\textbf{им. А.Н.Тихонова НИУ ВШЭ}
		
		\vspace{2ex}
		
		\textbf{Департамент компьютерной инженерии}
	\end{center}
	\vspace{1ex}	
	
	\begin{center}
		Курс «Системное проектирование цифровых устройств»
	\end{center}	
	
	
	\begin{center}
	\textbf{ОТЧЕТ\\
		ПО ЛАБОРАТОРНОЙ РАБОТЕ №1
	}
	\end{center}	

	\begin{center}
		Тема работы: «Разработка и программирование Soft-процессорных ядер с архитектурой однотактный MIPS. Часть 1»
	\end{center}

	\vspace{2ex}

	\begin{flushright}
		\textbf{Выполнили:}
		
		\vspace{2ex}
		
		Студенты группы БИВ174
		
		Бригада №5
		
		\vspace{2ex}
		
		Подчезерцев Алексей Евгеньевич
		
		Солодянкин Андрей Александрович
		\vspace{2ex}
		
		\textbf{Принял:}
		
		асс. МИЭМ НИУ ВШЭ
		
		Американов А.А.
		
	\end{flushright}

	\vfill
	\begin{center}
		Москва \the\year \, г.
	\end{center}
	
\end{titlepage}
\addtocounter{page}{1}
\section{Мотивационное письмо}
	
Заканчивая высшее образование по специальности «Информатика и вычислительная техника» в Высшей школе экономики (ВШЭ) факультет Московский институт электроники и математики (МИЭМ), я подаю документы на программу магистратуры в Сколковский институт науки и технологий, чтобы получить углубленные знания и навыки в сфере машинного обучения.
Машинное обучение – это одно активно развивающаяся IT отрасль в мире, я выбрал «Интеллектуальный анализ данных», как майнор в бакалавриате.

Моей специализацией в ВШЭ была Автоматизация проектирования.
В рамках этой специализации мы проходили проектирование вычислительных систем на кристалле.
Также мы обучали нейросети на ПЛИС. 
Во время учёбы в ВШЭ я осознал, важность применения методов машинного обучения в повседневной жизни.

За время обучения в бакалавриате я стажировался в ПАО Сбербанк по программе Sberseasons по направлению МЛ.
Также я стажировался в ВТБ банке по программе IT юниор по направлению МЛ.
В настоящее время я работаю в штате в ВТБ банке по тому же направлению.
Я хотел бы дальше развиваться в направлении машинного обучения.
За время пребывания в компаниях я увидел важность МЛ в производственной сфере.

Во время обучения в бакалавриате я принимал участие в различных хакатонах от компаний: КРОК, Яндекс, ВТБ и другие.
Кроме этого я принимал участие в различных соревнованиях по машинному обучению, например: IDAO, Я.Профессионал, Высшая лига, различые соревнования на Kaggle и другие.
Участие в соревнованиях показало важность теоретических и практических знаний пр решении сложных задач. 

В целом, я считаю, что выбранная мной магистерская программа является уникальным связующим звеном между моим первым образованием и будущей карьерой.
Я уверен, что готов к учебе на этой программе и сделаю все от меня зависящее, чтобы стать хорошим студентом и полноправным членом студенческого сообщества Сколтеха.
	
\end{document} % конец документа
